\documentclass{ssiBio}
\usepackage{siunitx}
\usepackage{verbatim}
\title{BstXI Restriction Digestion Protocol}
\author{Written by \textbf{Alan Tomusiak}}
\date{\textbf{Written:} May 20, 2019, \textbf{Printed:} \today{}}
\begin{document}

\maketitle
\section{Procedure Purpose}
Perform BstXI restriction digestion on AD template DNA in order to create a free 3' overhang. This overhang can be used for basepair addition in order to test the backspace method.

\section{Safety Information}
\begin{safety}
  \begin{enumerate}
    \SYBRGOLD
  \end{enumerate}
\end{safety}

\section{Materials}
\begin{itemize}
  \item{AD Template}
  \item{BstXI (10000 units/mL)}
  \item{10X NEBuffer 3.1}
  \item{Nuclease-free Water}
  \item{Agarose}
  \item{1X Tris-Borate-EDTA (TBE) Buffer}
  \item{Thermocycler}
  \item{Gel-running apparatus.}
  \item{PCR 8-strip tubes OR 96-well plate.}
  \item{10,000X Sybr Green I}
\end{itemize}

\section{Procedure}
\begin{enumerate}
  \item{Mix the following reagents in either a PCR tube or a 96-well plate such that the final concentrations match the following:}
  \begin{enumerate}
    \item{1\ug{} AD template}
    \item{1X NEBuffer 3.1}
    \item{10 units BstXI}
  \end{enumerate}
  \item{Fill remaining volume with nuclease-free water.}
  \helpfulTip{You can, and should, create a large master mix with enough materials for multiple reactions and then aliquot smaller volumes into either PCR tubes or 96-well plate wells.}
  \item{Incubate at 37\C{} for twenty minutes, and then at 80\C{} for 20 minutes.}
  \stopPoint
  \item{Validate successful restriction digestion utilizing a 2\% agarose gel.}
  \item{If necessary, further validate successful restriction digestion by sending in a purified sample to be Sanger sequenced.}
\end{enumerate}

\bibliographystyle{ieeetr}
\bibliography{main}
\end{document}
